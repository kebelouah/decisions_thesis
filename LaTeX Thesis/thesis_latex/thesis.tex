\documentclass[diploma]{softlab-thesis}

%%%
%%%  Add and configure the packages that you need for your thesis
%%%

\usepackage{minted}
\usepackage{Times New Roman}
\usepackage{url}
\usepackage{breakurl}
\sloppy

% Hyperref needs to be the last package to use!
\PassOptionsToPackage{hyphens}{url}
\usepackage[xetex,%
    pdfpagemode=UseOutlines,pdfstartview=FitW,%
    colorlinks=true,linkcolor=blue,citecolor=blue,urlcolor=blue,%
    bookmarks=true,bookmarksopen=true,bookmarksnumbered=false,%
    pdfencoding=auto,unicode=true,hyperfootnotes=true,%
    h
    ypertexnames=false,%
  ]{hyperref}


%%%
%%%  The document
%%%

\begin{document}

%%%  Title page

\frontmatter

\title{Πρόβλεψη Δικαστικών Αποφάσεων}
\author{Θεόδωρος Άγγελος Μέξης}
\authoren{Theodoros Angelos Mexis}
\date{Φεβρουάριος 2025 }
\datedefense{15}{2}{2025}

\supervisor{Παναγιώτης Τσανάκας}
\supervisorpos{Καθηγητής Ε.Μ.Π.}

\committeeone{Νικόλαος Σ. Παπασπύρου}
\committeeonepos{Καθηγητής Ε.Μ.Π.}
\committeetwo{Πέτρος Παπαδόπουλος}
\committeetwopos{Επίκ. Καθηγητής Ε.Μ.Π.}
\committeethree{Γεώργιος Νικολάου}
\committeethreepos{Αν. Καθηγητής Ε.Κ.Π.Α.}

\TRnumber{CSD-SW-TR-42-17}  % number-year, ask nickie for the number
\department{Τομέας Τεχνολογίας Πληροφορικής και Υπολογιστών}

\maketitle


%%%  Abstract, in Greek

\begin{abstractgr}%

\begin{keywordsgr}
  Μηχανική Μάθηση,
  Τεχνητή Νοημοσύνη,
  Δικαστικές Αποφάσεις,
  Πρόβλεψη
\end{keywordsgr}
\end{abstractgr}


%%%  Abstract, in English

\begin{abstracten}%

\begin{keywordsen}

\end{keywordsen}
\end{abstracten}


%%%  Acknowledgements

\begin{acknowledgementsgr}
 Θα ήθελα να ευχαριστήσω τον επιβλέποντα καθηγητή μου Παναγιώτη Τσανάκα για την ευκαιρία που μου δόθηκε να εργαστώ στο εξαιρετικά ενδιαφέρον θέμα της διπλωματικής μου εργασίας. Οφείλω ένα τεράστιο ευχαριστώ στον Δρ. Μάριο Κόνιαρη, ο οποίος μου προσέφερε πραγματικά, κάθε δυνατή βοήθεια κατά την εκπόνηση της διπλωματικής μου εργασίας.
 Θα ήθελα επίσης να ευχαριστήσω την οικογένειά μου, τους φίλους και συμφοιτητές μου, που ήταν πλάι μου σε όλη την διάρκεια της φοιτητικής μου ζωής.
 Αφιερώνω την εργασία αυτή στον αδελφό μου Κωνσταντίνο.
\end{acknowledgementsgr}

\begin{acknowledgementsen}

\end{acknowledgementsen}


%%%  Various tables

\tableofcontents
%\listoftables
\listoffigures
%\listofalgorithms


%%%  Main part of the book

\mainmatter

\chapter{Εισαγωγή}

Η νομοθεσία αναπτύσσεται και εξελίσσεται συνεχώς για να ανταποκριθεί στις νέες και μεταβαλλόμενες ανάγκες των κοινωνιών, αντιδρώντας στις κοινωνικές, πολιτικές, οικονομικές και τεχνολογικές αλλαγές. Οι συνεχείς μεταβολές στη νομοθεσία και η ραγδαία αύξηση του αριθμού των δεδικασμένων υποθέσεων προσθέτουν ένα ολοένα αυξανόμενο βάρος στους νομικούς επαγγελματίες. Αυτό οδηγεί φυσικά στο ερώτημα αν μπορεί να παρασχεθεί κάποια μηχανική υποστήριξη στον τομέα αυτό. Ένα τέτοιο σύστημα θα διευκόλυνε τη δουλειά δικηγόρων, εισαγγελέων και δικαστών, καθώς και άλλων συναφών επαγγελματιών, και θα μπορούσε να έχει θετική συμβολή στο δημόσιο συμφέρον εξοικονομώντας χρόνο, μειώνοντας τα λάθη και βελτιώνοντας τη συνέπεια των δικαστικών αποφάσεων. Οι υπολογιστές μπορούν να αναλύσουν τεράστια σύνολα δεδομένα νομικών κειμένων.

\subsection{Νομικά Κείμενα}

Τα νομικά κείμενα είναι κείμενα τα οποία έχουν συνταχθεί για διάφορους σκοπούς, με βασικό τους χαρακτηριστικό ότι σχετίζονται με τον νόμο είτε λόγω της ιδιότητας του συντάκτη (π.χ δικαστής), είτε λόγω των αναφορών που περιέχουν σε άλλα νομικά κείμενα, ή λόγω της θεματολογίας τους που αφορά την ρύθμιση των δικαιωμάτων και των υποχρεώσεων ιδιωτών και θεσμών. Ορισμένοι από τους βασικούς τύπους νομικών κειμέναν, ιεραρχημένοι βάσει της ισχύος τους ως πηγών δικαίου, είναι οι παρακάτω :

\begin{enumerate}
\item \textbf{Συντάγματα} : τα οποία αποτελούν τις θεμελιώδεις ραχές που αφορούν πως διοικείται ένα κράτος.
\item \textbf{Νομοθεσία} : η οποία περιλαμβάνει τους νόμους τους οποίους θεσπίζει ένα νομοθετικό σώμα και ρυθμίζουν τι είναι επιτρεπτό από τον νόμο. Συνήθως, οι νόμοι οργανώνονται θεματικά σε κώδικες παρόμοιας θεματολογίας (π.χ Ποινικός Κώδικας).
\item \textbf{Δικαστικές Αποφάσεις} : οι οποίες περιλαμβάνουν τα ενδιάμεσα ή τελικά αποτελέσματα μιας δίκης, την κρίση του δικαστηρίου για τα γεγονότα και τα επιχειρήματα της κάθε πλευράς καθώς και την απόφαση του δικαστηρίου σε γραπτή μορφή.
\item \textbf{Συμβόλαια} : τα οποία συνιστούν αμοιβαίες συμφωνίες μεταξύ συμβαλλόμενων μερών, με σκοπό να τηρηθούν αμοιβαίες υποχρεώσεις.
\end{enumerate}

Τα νομικά κείμενα έχουν εξελιχθεί σημαντικά μέσα στις χιλιετίες που παρήλθαν από την πρώτη χρήση τους. Τα πρώτα κείμενα ιδιωτικού δικαίου ήταν συμβόλαια, διαθήκες και νομικές πράξεις εγγεγραμμένες σε πινακίδες από πηλό στην Σουμερία περίπου 5000 χρόνια πριν. Παρομοίως, τα πρώτα κείμενα δημοσίου δικαίου, όπως νόμοι, εμφανίστηκαν στην Μεσοποταμία µε τους νόμους του βασιλιά Ουρ Ναμμού και αργότερα τον κώδικα του Χαμουραμπί να αποτελούν γνωστά παραδείγματα. 

Στην Αρχαία Ελλάδα, οι νομικές ρυθμίσεις που αφορούσαν ιδιωτικές υποθέσεις, όπως θέματα
κληρονομιάς, εμπορίου και συμβολαίων, διακρίνονταν από τις διατάξεις που ρύθμιζαν τη ζωή των πολιτών σε κάθε πόλη-κράτος. Η νομοθεσία και οι νόρμες διέφεραν ανά περιοχή, κυρίως λόγω της επιρροής της ρητορικής τέχνης, της διάδοσης της γνώσης των νόμων και της λειτουργίας των δικαστηρίων, καθώς και άλλων κοινωνικοπολιτικών παραμέτρων που καθόριζαν την ιδιότητα του πολίτη. Εμβληματικά παραδείγματα περιλαμβάνουν τους αυστηρούς νόμους του Δράκοντα (620 π.Χ.), οι οποίοι αργότερα μεταρρυθμίστηκαν από τον Σόλωνα (593 π.Χ.), προσδίδοντας μια πιο ισορροπημένη προσέγγιση στη νομοθεσία.



\chapter{Θεωρητικό υπόβαθρο}

\section{Τεχνικές Προβλέψεων}

\subsection{Ορισμός και Διαδικασία Πρόβλεψης}

Ως πρόβλεψη μπορεί να οριστεί η εκτίμηση αβέβαιων μελλοντικών γεγονότων. Οι προβλέψεις μπορούν να γίνουν βασισμένες στην εμπειρία και την παρατήρηση, σε στατιστικές μεθόδους, καθώς και σε πολύπλοκα μαθηματικά μοντέλα. Χρησιμοποιούνται για τη βελτίωση της λήψης αποφάσεων και σχεδιασμού. 

Η διαδικασία παραγωγής προβλέψεων είναι μια απαιτητική διαδικασία. Στην ακόλουθη παράγραφο θα περιγραφούν επιγραμματικά τα πέντε βασικά βήματα που είναι απαραίτητα για την παραγωγή και αξιολόγηση προβλέψεων:

\begin{enumerate}
    \item \textit{Καθορισμός του προβλήματος.} Συνιστά ένα από τα πιο σημαντικά και ταυτόχρονα δυσκολότερα μέρη της διαδικασίας παραγωγής προβλέψεων. Σε αυτό το βήμα γίνεται απόπειρα να καθοριστούν τα επιθυμητά μεγέθη που πρόκειται να προβλευθούν, καθώς και η μετέπειτα χρήση των προβλέψεων αυτών.
    \item \textit{Συλλογή των δεδομένων.} Η διαδικασία αυτή αποδεικνύεται συχνά χρονοβόρα, καθώς εκτός των μετρήσιμων αριθμητικών δεδομένων, σημαντική αποδεικνύεται και η χρήση διαθέσιμων εμπειρικών πληροφοριών για το αντικείμενο προς μελέτη. 
    \item \textit{Προεπεξεργασία των δεδομένων.} Ένα καίριο βήμα για την παραγωγή προβλέψεων συνιστά η απόκτηση μιας ολοκληρωμένης αίσθησης των διαθέσιμων δεδομένων, έτσι ώστε να εντοπιστούν πιθανά λάθη, ασυνήθιστες τιμές, σημαντικές τάσεις ή εποχικότητα. Σκοπός της προεπεξεργασίας των δεδομένων είναι η δημιουργία ενός εξομαλυμένου συνόλου δεδομένων για την εφαρμογή των μοντέλων πρόβλεψης.
     \item \textit{Επιλογή μεθόδων πρόβλεψης.} Επιτυγχάνεται η ορθή επιλογή μοντέλων πρόβλεψης καθώς και η ιδιαίτερα σημαντική διαδικασία επιλογής των κατάλληλων παραμέτρων τους, ώστε να παραχθούν τα πλέον ακριβή αποτελέσματα. 
      \item \textit{Χρήση και αξιολόγηση των μοντέλων πρόβλεψης.} Το τελικό στάδιο περιλαμβάνει την χρήση των επιλεγμένων μοντέλων ώστε να παραχθούν οι ζητούμενες προβλέψεις. Το κατά πόσο οι προβλέψεις των επιλεγμένων μοντέλων είναι ικανοποιητικές μπορεί να κριθεί μόνο με την πάροδο του χρόνου, και πιο συγκεκριμένα καθώς τα νέα δεδομένα γίνονται διαθέσιμα. Η αξιολόγηση και η μέτρηση της ακρίβειας των προβλέψεων επιτυγχάνεται με εξειδικευμένους στατιστικούς δείκτες.
\end{enumerate}

\subsection{Κατηγορίες Μεθόδων Πρόβλεψης}

Οι μέθοδοι πρόβλεψης, σύμφωνα με την διαδικασία παραγωγής τους, διακρίνονται σε τρεις μεγάλες κατηγορίες :
\begin{enumerate}
\item \textbf{Ποσοτικές Μέθοδοι}. Οι ποσοτικές μέθοδοι αναφέρονται στην εφαρμογή στατιστικών μοντέλων χρονοσειρών ή αιτιοκρατικών μοντέλως επί μιας σειρά δεδομένων με σκοπό αυτοματοποιημένη και συστηματική παραγωγή προβλέψεων. Οι στατιστικές προβλέψεις είναι αποδεκτά ακριβείς και εφαρμόσιμες, αν συνδυαστούν με κατάλληλα διαστήματα εμπιστοσύνης. Προϋποθέτουν ότι η συμπεριφορά της εκάστοτε χρονοσειράς θα συνεχιστεί στο μέλλον, κάτι το οποίο δεν συμβαίνει πάντα. Επιπροσθέτως, κύρια παραδοχή των μοντέλων αυτών συνιστά η σταθερή συσχέτιση μεταξύ του προς πρόβλεψη μεγέθους και άλλων παραγόντων, χωρίς ωστόσο να είναι απαραίτητ η ύπαρξη χρονική εξάρτησης. Η συλλογή των δεδομένων αποτελεί συχνά μια χρονοβόρα και ενίοτε δύσκολη διαδικασία, καθώς απαιτείται μεγ
άλο πλήθος ιστορικών δεδομένων προκειμένουν να παραχθούν οι ζητούμενες προβλέψεις. Τέτοια μοντέλα είναι οι μέθοδοι εκθετικής εξομάλυνσης, τα μοντέλα παλινδρόμησης, τα μοντέλα ARIMA και τα τεχνητά νευρωνικά δίκτυα. 
\item \textbf{Κριτικές Μέθοδοι}. Οι κριτικές μέθοδοι πρόβλεψης δεν έχουν τις ίδιες απαιτήσεις σε
δεδομένα όπως οι στατιστικές μέθοδοι. Τα δεδομένα των κριτικών μεθόδων αποτελούν προϊόν διαίσθησης, κρίσης και συσσωρευμένης γνώσης από πλευράς εμπειρογνωμόνων. Οι μέθοδοι αυτές μπορούν να λάβουν υπόψη ειδικά γεγονότα και ενέργειες, ενώ ταυτόχρονα έχουν τη δυνατότητα να αντισταθμίζουν ανεπάρκειες και ελλείψεις σε ιστορικά δεδομένα. Είναι κατάλληλες όταν θίγονται ηθικά ζητήματα που υπερισχύουν των οικονομικών και τεχνολογικών παραγόντων. Οι μέθοδοι αυτές πρέπει να λειτουργούν συμπληρωματικά με τις μεθόδους στατιστικής μελέτης. Ανάμεσα στις πιο διαδεδομένες μεθόδους συγκαταλέγονται η απλή κρίση, η μέθοδος Delphi και οι δομημένες αναλογίες.
\item \textbf{Κριτικές Μέθοδοι}. Οι τεχνολογικές μέθοδοι πρόβλεψης αφορούν κυρίως μακροπρόθεσμα πλάνα τεχνολογικής, οικονομικής, κοινωνικής και πολιτικής φύσης. Διακρίνονται σε διερευνητικές και κανονιστικές. Οι πρώτες έχουν ως σημείο εκκίνησης το παρελθόν και το παρόν και στοχεύουν στη διερεύνηση όλων των πιθανών μελλοντικών περιπτώσεων. Οι κανονιστικές έχουν προκαθορισμένους στόχους και εξετάζουν τη δυνατότητα επίτευξής τους, σύμφωνα με τους υπάρχοντες περιορισμούς και διαθέσιμους πόρους [Φ13].
\end{enumerate}

\subsection{Μηχανική Μάθηση}

Η μηχανική μάθηση είναι υποπεδίο της επιστήμης των υπολογιστών που αναπτύχθηκε από τη μελέτη της αναγνώρισης προτύπων και της υπολογιστικής θεωρίας μάθησης στην τεχνητή νοημοσύνη. Η μηχανική μάθηση διερευνά τη μελέτη και την κατασκευή αλγορίθμων που μπορούν να μαθαίνουν από τα δεδομένα και να κάνουν προβλέψεις σχετικά με αυτά. Οι αλγόριθμοι αυτοί βελτιώνουν τη συμπεριφορά τους σε κάποια εργασία χρησιμοποιώντας την εμπειρία τους. Τέτοιοι αλγόριθμοι λειτουργούν κατασκευάζοντας μοντέλα από πειραματικά δεδομένα, προκειμένου να κάνουν προβλέψεις βασιζόμενες στα δεδομένα ή να εξάγουν αποφάσεις που εκφράζονται ως το αποτέλεσμα. Ο Άρθουρ Σάμουελ ορίζει τη μηχανική μάθηση ως \textit{"Πεδίο μελέτης που δίνει στους υπολογιστές την ικανότητα να μαθαίνουν, χωρίς να έχουν ρητά προγραμματιστεί"}.

Ο τομέας της μηχανικής μάθησης αναπτύσσει τρεις τρόπους μάθησης, ανάλογους με
τους τρόπους με τους οποίους μαθαίνει ο άνθρωπος:
\begin{enumerate}
\item \textbf{Επιβλεπόμενη μάθηση} \textit{(Supervised Learning)}. Η επιβλεπόμενη μάθηση είναι η διαδικασία όπου ο αλγόριθμος κατασκευάζει μια συνάρτηση που απεικονίζει δεδομένες εισόδους
(σύνολο εκπαίδευσης) σε γνωστές επιθυμητές εξόδους, με απώτερο στόχο τη γενίκευση της
συνάρτησης αυτής και για εισόδους με άγνωστη έξοδο. Χρησιμοποιείται σε προβλήματα
ταξινόμησης (classification), πρόγνωσης (prediction) και διερμηνείας (interpretation).
\item \textbf{Μη επιβλεπόμενη μάθηση} \textit{(Unsupervised Learning)}. Στην μη επιβλεπόμενη μάθηση ο
αλγόριθμος κατασκευάζει ένα μοντέλο για κάποιο σύνολο εισόδων υπό μορφή
παρατηρήσεων χωρίς να γνωρίζει τις επιθυμητές εξόδους. Χρησιμοποιείται σε προβλήματα
ανάλυσης συσχετισμών (association analysis) και ομαδοποίησης (clustering).
\item \textbf{Ενισχυτική μάθηση}
 \textit{(Reinforcement Learning)}. Στην ενισχυτική μάθηση ο αλγόριθμος
μαθαίνει μια στρατηγική ενεργειών μέσα από άμεση αλληλεπίδραση με το περιβάλλον.
Χρησιμοποιείται κυρίως σε προβλήματα σχεδιασμού, όπως ο έλεγχος κίνησης ρομπότ και η
βελτιστοποίηση εργασιών σε εργοστασιακούς χώρους.
\end{enumerate}

\section{Μοντέλα Πρόβλεψης - Ταξινόμησης}

Οι πιο γνωστές και χρήσιμες μέθοδοι για την πρόβλεψη έκβασης δικαστικών αποφάσεων -βάσει της βιβλιογραφίας- είναι τα Δέντρα Αποφάσεων (\textit{Decision Trees}), τα Τυχαία Δάση (\textit{Random  Forest}), οι Μηχανές Υποστήριξης Διανυσμάτων ((\textit{SVMs}) καθώς και η Γραμμική Παλινδρόμηση (\textit{Linear Regression}). 

\subsection{Τυχαία Δάση - Random Forests}

\subsection{Δέντρα Αποφάσεων - Decision Trees}

\subsection{Μηχανές Υποστήριξης Διανυσμάτων - SVM}

\subsection{Γραμμική Παλινδρόμηση - Linear Regression}


\chapter{Παρουσίαση Συνόλου Δεδομένων}

\sloppy
Οι δικαστικές αποφάσεις που χρησιμοποιήθηκαν στην παρούσα εργασία συλλέχθηκαν από το Εφετείο Πειραιώς \href{https://www.efeteio-peir.gr/?page_id=4017}{(https://www.efeteio-peir.gr/?page_id=4017)} και από τον Άρειο Πάγο \href{https://www.areiospagos.gr/nomologia/apofaseis.asp}{(https://www.areiospagos.gr/nomologia/apofaseis.asp)}. 
Πρόκειται για αποφάσεις που ελήφθησαν κατά τα έτη 2009, 2018, 2021 και 2022 από τα συγκεκριμένα δικαστήρια και καλύπτουν διάφορους τομείς του δικαίου. Τόσο η προεπεξεργασία των δεδομένων αλλά και η επισημείωσή τους ήταν απαραίτητες διαδικασίες προκειμένου να δημιουργηθεί η τελική μορφή του συνόλου δεδομένων προς μελέτη. Οι λεπτομέρειες των διαδικασιών αυτών θα αναλυθούν παρακάτω.



\section{Προεπεξεργασία Δεδομένων}

Ο πρωταρχικός στόχος της προεπεξεργασίας των δεδομένων είναι να προετοιμάσουμε το κείμενο, αφαιρώντας περιττούς χαρακτήρες, αριθμούς, και αγγλικούς χαρακτήρες, έτσι ώστε να διευκολύνουμε την διαδικασία ανάλυσης και την επεξεργασία τους. Οι συγκεκριμένες ενέργειες είναι απαραίτητες με σκοπό να φέρουμε τις δικαστικές αποφάσεις σε μορφή κατάλληλη για τα μοντέλα που θα εξετάσουμε στην συνέχεια. Η διαδικασία προεπεξεργασίας των κειμένων είναι ένα κρίσιμο βήμα στην προετοιμασία των δεδομένων για τη χρήση τους σε αλγορίθμους μηχανικής μάθησης. Για την προεπεξεργασία των δικαστικών αποφάσεων, ακολουθήσαμε μια σειρά από βήματα που στοχεύουν στην αργότερα αποτελεσματική ανάλυση των κειμένων από τα μοντέλα.

\begin{enumerate}
\item \textit{Μετατροπή σε πεζά :} Όλα τα γράμματα μετατράπηκαν σε πεζά για να εξασφαλιστεί η συνέπεια και να αποφεύγεται η διάκριση μεταξύ κεφαλαίων και πεζών χαρακτήρων, που δεν θα πρόσθεταν κάποια αξία στην ανάλυση.
\item \textit{Αφαίρεση τονισμού :} Οι τόνοι αφαιρέθηκαν από τις λέξεις, διευκολύνοντας την ταύτιση όρων με και χωρίς τόνο, όπως «δικαστής» και «δικαστης», τα οποία θα αντιμετωπίζονταν ως διαφορετικές λέξεις από τον αλγόριθμο.
\item \textit{Αφαίρεση σημείων στίξης :} Τα σημεία στίξης αφαιρέθηκαν, καθώς δεν προσφέρουν πληροφορίες χρήσιμες για την εκπαίδευση των μοντέλων πρόβλεψης. Αυτό περιλαμβάνει όλα τα σημεία στίξης, όπως κόμματα, τελείες, ερωτηματικά κ.λπ.
\item \textit{Αφαίρεση αριθμών :} Οι αριθμοί αφαιρέθηκαν από τα κείμενα, καθώς σε πολλές περιπτώσεις δεν παρέχουν ουσιαστικές πληροφορίες για την ανάλυση, ιδιαίτερα όταν δεν συνδέονται με κρίσιμες πληροφορίες για το νόημα των αποφάσεων.
\item \textit{Αφαίρεση αγγλικών χαρακτήρων :} Επειδή οι δικαστικές αποφάσεις είναι στα ελληνικά, οποιοσδήποτε αγγλικός χαρακτήρας αφαιρέθηκε από τα δεδομένα.
\item \textit{Αφαίρεση ειδικών χαρακτήρων :} Αφαιρέθηκαν ειδικοί χαρακτήρες όπως η κάτω παύλα, που δεν προσθέτουν νόημα στο κείμενο και μπορεί να προκαλέσουν προβλήματα στη διαδικασία ανάλυσης.
\item \textit{Αφαίρεση λέξεων-κλειδιά :} Αφαιρέθηκαν λέξεις-κλειδιά, δηλαδή συχνές λέξεις οι οποίες δεν φέρουν σημαντική σημασιολογική πληροφορία, με χρήση της λίστας που παρέχεται από το NLTK \href{https://github.com/hb20007/hands-on-nltk-tutorial/blob/main/7-1-NLTK-with-the-Greek-Script.ipynb}{(https://github.com/hb20007/hands-on-nltk-tutorial/blob/main/7-1-NLTK-with-the-Greek-Script.ipynb)}.

\section{Διαδικασία Επισημείωσης}

Μετά την ολοκλήρωση της προεπεξεργασίας των κειμένων των δικαστικών αποφάσεων, προχωρήσαμε στην φάση της επισημείωσης των δεδομένων, προκειμένου να κατηγοριοποιήσουμεε της αποφάσεις δυαδικά, δηλαδή ως αποδοχή ή απόρριψη. Η επισημείωση είναι ένα κρίσιμο βήμα στη διαδικασία ανάλυσης δεδομένων, ιδιαίτερα όταν χρησιμοποιούνται τεχνικές μηχανικής μάθησης. Η ακρίβεια της επισημείωσης ενός συνόλου δεδομένων, επηρεάζει άμεσα την απόδοση των μοντέλων πρόβλεψης που θα εκπαιδευτούν πάνω σε αυτά τα δεδομένα. Στην περίπτωση των δικαστικών αποφάσεων, η σωστή ετικετοποίησης (\textit{labeling}) των δεδομένων είναι καθοριστική για την ανάπτυξη αξιόπιστων αλγορίθμων που μπορούν να βοηθήσουν στη βελτίωση της δικαστικής διαδικασίας. 
Η επισημείωση βασίστηκε στην ύπαρξη συγκεκριμένων φράσεων-κλειδιά  (\textit{regular expressions}) που χρησιμοποιούνται από τα δύο δικαστήρια, οι οποίες υποδηλώνουν την αποδοχή της αίτησης ή της έφεσης. Η διαδικασία που ακολουθήσαμε αναλύεται παρακάτω :

\begin{enumerate}
\item \textit{Ανάκτηση κειμένου :} Τα δεδομένα που χρησιμοποιήθηκαν προήλθαν από δικαστικές αποφάσεις αποθηκευμένες σε αρχεία PDF και HTML. Για την εξαγωγή του κειμένου από τα αρχεία HTML, χρησιμοποιήθηκε το εργαλείο BeautifulSoup, το οποίο επιτρέπει την ανάκτηση του πλήρους περιεχομένου των αποφάσεων. Αντίστοιχα, για τα αρχεία PDF και CSV, χρησιμοποιήθηκε η Python βιλβιοθήκη pandas προκειμένουν να γίνει η εξαγωγή του κειμένου αγνοώντας tags κι άλλες δομικές πληροφορίες που περιέχονται στα αρχεία. Αυτή η διαδικασία εξασφάλισε ότι το κείμενο εξάγεται με συνέπεια και ακρίβεια, ανεξάρτητα από την πηγή του.
\item \textit{Αναζήτηση στόχων - target words :} Προκειμένου να γίνει ορθή κατηγοριοποίηση των αποφάσεων που εξετάζουμε, καθορίσαμε μια λίστα από φράσεις-κλειδιά (\textit{regular expressions}). Πιο συγκεκριμένα, δημιουργήθηκε μια λίστα με φράσεις-κλειδιά, τα οποία χρησιμοποιούνται συνήθως σε αποφάσεις που καταλήγουν σε αποδοχή. Οι φράσεις αυτές, π.χ. «δέχεται τυπικά και κατ’ ουσίαν» ή «δέχεται τυπικά και ουσιαστικά», επιλέχθηκαν με βάση την ανάλυση της γλώσσας που χρησιμοποιείται στα δικαστικά κείμενα που εξετάζουμε και αντιστοιχούν σε περιπτώσεις όπου το δικαστήριο κάνει αποδεκτή την αίτηση ή την έφεση. Οι αποφάσεις που περιείχαν αυτές τις φράσεις επισημάνθηκαν ως αποδοχή (με την ένδειξη 0), ενώ οι υπόλοιπες επισημάνθηκαν ως απορρίψη (με την ένδειξη 1).
\item \textit{Δημιουργία συνόλου δεδομένων :} Μετά την αναζήτηση των λέξεων-στόχων, οι αποφάσεις επισημάνθηκαν κατάλληλα, και το αποτέλεσμα αποθηκεύτηκε σε ένα δομημένο σύνολο δεδομένων (CSV αρχείο), το οποίο περιέχει για κάθε απόφαση το όνομα του αρχείου και την αντίστοιχη κατηγορία στην οποία ανήκει.


Με το πέρας της διαδικασίας της επισημείωσης, το σύνολο των δικαστικών αποφάσεων είναι πλέον έτοιμο για την επόμενη φάση της μελέτης μας, όπου θα εφαρμοστούν τεχνικές μηχανικής μάθησης για την εξαγωγή προβλέψεων σχετικά με την έκβαση μελλοντικών υποθέσεων.

 
\chapter{Μέθοδοι Πρόβλεψης Δικαστικών Αποφάσεων}


\section{Προετοιμασία Κειμένων}

Προκειμένου να εξετάσουμε την αποτελεσματικότητα των διαφόρων ταξινομητών στο σύνολο των αποφάσεων που έχουμε στη διάθεσή μας ήταν απαραίτητο να αναπαραστίσουμε τα κείμενα των αποφάσεων σε αριθμητική μορφή.

Το TF-IDF (\textit{Term Frequency-Inverse Document Frequency}) είναι μία από τις πιο διαδεδομένες και αποτελεσματικές τεχνικές για την αναπαράσταση αυτή, στην Επεξεργασία Φυσικής Γλώσσας (\textit{NLP}). Το TF-IDF είναι ουσιαστικά μια αριθμητική στατιστική που προορίζεται να αντικατοπτρίζει τη σημασία μιας λέξης για ένα έγγραφο σε μια συλλογή ή ένα σώμα κειμένων. Πιο συγκεκριμένα, η τεχνική που εφαρμόσαμε στις αποφάσεις συνδυάζει δύο βασικές έννοιες: τη συχνότητα εμφάνισης μιας λέξης σε ένα έγγραφο (Term Frequency) και τη σπανιότητα αυτής της λέξης στο σύνολο των εγγράφων (Inverse Document Frequency).

Αναλυτικότερα, η συνάρτηση TF μετρά πόσες φορές μια λέξη εμφανίζεται σε ένα έγγραφο σε σχέση με το συνολικό αριθμό λέξεων, ενώ η συνάρτηση IDF μειώνει τη βαρύτητα των όρων που εμφανίζονται σε πολλά έγγραφα, καθώς αυτοί δεν είναι τόσο διακριτικοί. Ο πολλαπλασιασμός των δύο αυτών μεγεθών οδηγεί σε μια μετρική που αναδεικνύει τους πιο "σημαντικούς" όρους για κάθε έγγραφο. Το TF-IDF χρησιμοποιείται ευρέως σε συστήματα ανάκτησης πληροφορίας και ταξινομήσεις κειμένων.

Στην παρούσα διπλωματική εργασία χρησιμοποιείται το αντικείμενο Tf-idfVectorizer(), το οποίο δέχεται ως είσοδο κείμενο και εξάγει διανύσματα (vectors) σε ένα μοντέλο διανυσματικού χώρου. Το αντικείμενο αυτό ανάλογα με τα ορίσματα που δέχεται διαχειρίζεται και διαφορετικά τα δεδομένα. 

Οι μεταβλητές maxdf και mindf, που αποτελούν επίσης παραμέτρους του TfidfVectorizer(), παίζουν εξίσου σημαντικό ρόλο και βοηθούν στη μείωση των διαστάσεων κάθε μοντέλου, καθώς μπορούν ανάλογα με τις τιμές που θα λάβουν να περιορίσουν το εύρος του λεξιλογίου που δημιουργείται. Οι τιμές που χρησιμοποιήθηκαν σε αυτήν την εργασία επιλέχθηκαν εμπειρικά, έπειτα ωστόσο από πολλές δοκιμές. Όσον αφορά στο maxdf, με την τιμή 0,5 δηλώνεται ότι πρόκειται να αγνοηθούν όλοι οι όροι που εμφανίζονται σε πάνω από το 50 τοις εκατό των δεδομένων, ενώ σχετικά με την τιμή του mindf δηλώνεται ότι δεν θα ληφθούν υπόψη οι όροι που υπάρχουν σε λιγότερο από 10 έγγραφα.

\section{Εκπαίδευση και Ρύθμιση Υπερπαραμέτρων των Μοντέλων}

Στο παρόν κεφάλαιο, παρουσιάζεται η μεθοδολογία που ακολουθήθηκε για την εκπαίδευση των μοντέλων, καθώς και για την ρύθμιση των υπερπαραμέτρων τους (hyperparameter tuning).

\subsection{Βέλτιστη ρύθμιση των υπερπαραμέτρων}

Η σωστή λειτουργία πολλών από τους αλγορίθμους μηχανικής μάθησης, βασίζεται στη σωστή ρύθμιση των παραμέτρων τους. Οι υπερπαράμετροι (hyperparameters) συνιστούν τις μεταβλητές που χαρακτηρίζουν τη διαδικασία εκπαίδευσης του εκάστοτε μοντέλου. Οι τιμές τους πρέπει να ρυθμιστούν από τον προγραμματιστή προτού αρχίσει η διαδικασία εκπαίδευσης, εν αντιθέσει με τις απλές παραμέτρους του μοντέλου, η τιμή των οποίων υπολογίζεται αυτομάτως -κατά την εκπαίδευση- από το ίδιο το μοντέλο. Ως εκ τούτου, γίνεται κατανοητή η σημασία της επιλογής των κατάλληλων υπερπαραμέτρων για την αποδοτική λειτουργία κάθε μοντέλου. Ωστόσο, δεν υπάρχουν σαφείς και ακριβείς κανόνες που να προσδιορίζουν τον τρόπο με τον οποίο πρέπει να γίνει αυτή η επιλογή. Οι αλγόριθμοι που επιτελούν την ρύθμιση των υπερπαραμέτρων λειτουργούν σύμφωνα με τη μέθοδο δοκιμής-σφάλματος, προβαίνοντας σε συνεχόμενες επιλογές τιμών, εως ότου φτάσουν στις βέλτιστες τιμές. Επομένως, σημαντικό βήμα αποτελεί η επιλογή των τιμών εκείνων που θα υποβληθούν σε αυτή τη διαδικασία βελτιστοποίησης.

Υπάρχουν ελάχιστες καθολικές συμβουλές σχετικά με την επιλογή των τιμών αυτών, ενώ η τελική επιτυχία της
διαδικασίας εξαρτάται σε μεγάλο βαθμό από την εμπειρία του προγραμματιστή. Η επιλογή τους πρέπει να γίνεται με σύνεση, καθώς κάθε παράμετρος που επιλέγεται να ρυθμιστεί μπορεί να αυξήσει εκθετικά τον απαιτούμενο αριθμό των δοκιμών. Αφού επιλεχθούν οι παράμετροι προς ρύθμιση, εφαρμόζονται αλγόριθμοι, οι οποίοι προελαύνουν το χώρο αναζήτησης που 85 δημιουργείται, το μέγεθος του οποίου εξαρτάται από το πλήθος και το εύρος των υπερπαραμέτρων που έχουν επιλεχθεί να ρυθμιστούν. Οι δύο πλέον χρησιμοποιούμενοι από τους αλγορίθμους αυτούς είναι οι : 

\begin{enumerate}
\item \textbf{Αναζήτηση Πλέγματος} (\textit{Grid Search}). Η αναζήτηση πλέγματος αποτελεί τον απλούστερο αλγόριθμο βελτιστοποίησης υπερπαραμέτρων. Ο αλγόριθμος αυτός εκτελεί μια εξαντλητική αναζήτηση στο προκαθορισμένο χώρο αναζήτησης που δημιουργείται. Ο χώρος αναζήτησης μπορεί να καταλήξει να αποτελεί ένα υπερεπίπεδο δεκάδων διαστάσεων, ανάλογα με το πλήθος των προς ρύθμιση παραμέτρων. 
\item \textbf{Τυχαία Αναζήτηση} (\textit{Ranodmized Search}). Στην τυχαία αναζήτηση, ο χώρος αναζήτησης διασχίζεται τυχαία έως ότου ικανοποιηθεί κάποιο κριτήριο τερματισμού, όπως ο αριθμός των επαναλήψεων. Ο αλγόριθμος αυτός δεν εγγυάται την εύρεση της βέλτιστης λύσης, αλλά λειτουργεί ικανοποιητικά σε προβλήματα που το πλήθος των υπερπαραμέτρων είναι μικρό, ενώ επιλέγεται επίσης, όταν δεν είναι διαθέσιμη μεγάλη υπολογιστική ισχύς.
\end{enumerate}

Ένας κύριος λόγος μη αποδοτικότητας ενός μοντέλου μηχανικής μάθησης είναι η υπερπροσαρμογή (\textit{overfitting}). Ο συγκεκριμένος όρος χρησιμοποιείται στην επιβλεπόμενη μάθηση για να δηλώσει την κατάσταση κατά την οποία ένα μοντέλο έχει εκπαιδευτεί και εξειδικευτεί στο σύνολο εκπαίδευσης του προβλήματος που εξετάζεται, με αποτέλεσμα να παρουσιάζει χαμηλή ακρίβεια στην πρόβλεψη του συνόλου δοκιμής. 

Προκειμένου να αντιμετωπιστεί το παραπάνω πρόβλημα, χρησιμοποιήθηκε η τεχνική \textit{Cross-Validation}(CV), η οποία αποτελεί την πλέον ενδεικνυόμενη λύση. Με την τεχνική αυτή, δεν απαιτείται πλέον η δέσμευση ενός μέρους του συνόλου εκπαίδευσης σα σύνολο αξιολόγησης, με αποτέλεσμα τα μοντέλα να εκπαιδεύονταιμε το μέγιστο δυνατό αριθμό δειγμάτων. Ωστόσο, το σύνολο δοκιμής εξακολουθεί να υπάρχει για την τελική αξιολόγηση των μοντέλων. Η βασική πρακτική της εν λόγω τεχνικής ονομάζεται \textit{k-fold Cross-Validation}. Πιο συγκεκριμένα, επιλέγεται ένας σταθερός αριθμός από \textit{folds (πτυχές)}, δηλαδή συνεχόμενες διαιρέσεις των δεδομένων. Τα δεδομένα διαχωρίζονται σε \textit{k} προσεγγιστικά ίσα folds και κάθε ένα στη συνέχεια θα χρησιμοποιηθεί επαναληπτικά για την αξιολόγηση, ενώ τα υπόλοιπα για την εκπαίδευση των μοντέλων. Τα k−1 folds χρησιμοποιούνται ως σύνολο εκπαίδευσης, ενώ το 1 fold λειτουργεί ως σύνολο αξιολόγησης. Η διαδικασία αυτή επαναλαμβάνεται συνολικά  k φορές, διασφαλίζοντας ότι κάθε δείγμα του συνόλου δεδομένων αξιολογείται μία φορά και συμμετέχει k−1 φορές στο σύνολο εκπαίδευσης. Τυπικές τιμές του k είναι της τάξεως του 5 έως 10. Η συνολική αξιολόγηση του μοντέλου προκύπτει από τη μέση τιμή των επιμέρους αξιολογήσεων που προέκυψαν κατά τις k επαναλήψεις. Η διαδικασία αυτή, μπορεί να επαναληφθεί για κάθε τιμή των υπερπαραμέτρων που δοκιμάζονται, ώστε να επιλεχθούν τελικά οι βέλτιστες. Είναι σαφές ότι η πρακτική αυτή έχει μεγάλο υπολογιστικό κόστος, καθώς απαιτούνται πολλοί κύκλοι εκπαίδευσης του μοντέλου. Ωστόσο, η σημασία της έγκειται στο γεγονός ότι δε δεσμεύει μεγάλο μέρος των διαθέσιμων διεγμάτων προς αξιολόγηση, γεγονός θεμελιώδους σημασίας όταν ο αριθμός των δειγμάτων είναι περιορισμένος. 

Στην παρούσα εργασία, τόσο η μέθοδος Grid Search όσο και η Randomized Search χρησιμοποιήθηκαν μέσω των σχετικών συναρτήσεων της βιβλιοθήκης Scikit-learn, για την βέλτιστη ρύθμιση των υπερπαραμέτρων των μοντέλων. Η δεύτερη μέθοδος χρησιμοποιήθηκε, λόγω του απαγορευτικά υψηλού υπολογιστικού χρόνου που απαιτούταν, σε περιπτώσεις μοντέλων με μεγάλο αριθμό υπερπαραμέτρων και πολλές υπό δοκιμή τιμές. Οι δύο αυτοί αλγόριθμοι, δέχονται σαν ορίσματα το μοντέλο πρόβλεψης, τα σύνολα τιμών των υπερπαραμέτρων που θα δοκιμαστούν, την τεχνική Cross-Validation που θα εφαρμοστεί και τον τρόπο με τον οποίο θα γίνει η αξιολόγηση (scoring) του μοντέλου. Στη συνέχεια, για κάθε δυνατό συνδυασμό των υπερπαραμέτρων του ορίσματος, εκτελείται η σχετική τεχνική Cross-Validation και προκύπτει η αξιολόγηση του εκάστοτε μοντέλου. Τέλος, ο αλγόριθμος επιλέγει το μοντέλο με εκείνες τις υπερπαραμέτρους που έδωσαν την υψηλότερη απόδοση στο σύνολο αξιολόγησης της τεχνικής Cross-Validation.



\chapter{Πειραματικά Αποτελέσματα - Ερμηνεία}

\chapter{Επίλογος}



\selectlanguage{greek}


%%%  Bibliography

% You shouldn't want to include all the contents of thesis.bib
% in your bibliography (do you?)
\nocite{*}

\bibliographystyle{softlab-thesis}
\bibliography{thesis}


%%%  End of document

\end{document}
